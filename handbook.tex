\documentclass[11pt,a4paper]{article}
\usepackage[utf8]{inputenc}  % Added to handle UTF-8 encoding
\usepackage[T1]{fontenc}     % Added for better font encoding
\usepackage[margin=1in]{geometry}
\usepackage[theorems,algorithms]{textool}  % Load with all features for documentation
\usepackage{listings}
\usepackage{tcolorbox}
\usepackage{tabularx}
\usepackage{booktabs}
\usepackage{fancyhdr}
\usepackage[toc,page]{appendix}

% Setup for code listings
\lstset{
	language=[LaTeX]TeX,
	basicstyle=\ttfamily\small,
	commentstyle=\color{green!50!black},
	keywordstyle=\color{blue},
	breaklines=true,
	frame=single,
	backgroundcolor=\color{gray!10},
	numbers=left,
	numberstyle=\tiny\color{gray},
	escapeinside={(*}{*)}
}

% Define example environment
\newtcolorbox{example}{
	colback=blue!5!white,
	colframe=blue!75!black,
	title=Example,
	fonttitle=\bfseries
}

\newtcolorbox{warning}{
	colback=red!5!white,
	colframe=red!75!black,
	title=Warning,
	fonttitle=\bfseries
}

\newtcolorbox{tip}{
	colback=green!5!white,
	colframe=green!75!black,
	title=Tip,
	fonttitle=\bfseries
}

\newtcolorbox{important}{
	colback=orange!5!white,
	colframe=orange!75!black,
	title=Important,
	fonttitle=\bfseries
}

% Header and footer
\pagestyle{fancy}
\fancyhf{}
\fancyhead[L]{\textit{TexTool Package Handbook}}
\fancyhead[R]{\textit{Version 4.0}}
\fancyfoot[C]{\thepage}

\title{\Huge\bfseries TexTool Package Handbook\\[0.5em]
	\Large A Comprehensive LaTeX Package for\\Mathematical Typesetting and Figure Management\\[0.5em]
	\normalsize Version 4.0 - Robust Edition}
\author{D.W.\\
	\texttt{wdong025@ucr.edu}}
\date{November 2025}

\begin{document}
	
	\maketitle
	\thispagestyle{empty}
	\newpage
	
	\tableofcontents
	\newpage
	
	% ============================================
	\section{Introduction}
	% ============================================
	
	The \texttt{textool} package is a comprehensive LaTeX package designed to streamline academic writing, particularly in mathematics and technical fields. It provides:
	
	\begin{itemize}
		\item Extensive mathematical symbol libraries with consistent naming conventions
		\item Simplified figure and multi-figure insertion commands
		\item Optional theorem environments (opt-in)
		\item Optional algorithm support (opt-in)
		\item Enhanced equation formatting tools
		\item Probability and statistical operators
		\item Color definitions for presentations
		\item Robust compatibility with other packages
	\end{itemize}
	
	\subsection{What's New in Version 4.0}
	
	\begin{important}
		Version 4.0 introduces significant changes to improve compatibility and stability:
		\begin{itemize}
			\item \textbf{Opt-in model}: Theorems and algorithms are now OFF by default
			\item \textbf{Enhanced compatibility}: Better conflict resolution with other packages
			\item \textbf{Tabularx support}: Fixed array package loading for tabularx compatibility
			\item \textbf{Robust loading}: Conditional checks prevent redefinition conflicts
		\end{itemize}
	\end{important}
	
	\subsection{Installation}
	
	Place \texttt{textool.sty} in your LaTeX project directory or in your local \texttt{texmf} tree. Then load it in your document preamble with the desired options:
	
	\begin{lstlisting}
		% Basic usage (math symbols and figures only)
		\usepackage{textool}
		
		% With theorem environments
		\usepackage[theorems]{textool}
		
		% With algorithm support
		\usepackage[algorithms]{textool}
		
		% With both theorems and algorithms
		\usepackage[theorems,algorithms]{textool}
		
		% Minimal mode (math only, no graphics)
		\usepackage[minimal]{textool}
	\end{lstlisting}
	
	\subsection{Package Options}
	
	\begin{table}[h]
		\centering
		\begin{tabularx}{\textwidth}{llX}
			\toprule
			\textbf{Option} & \textbf{Default} & \textbf{Description} \\
			\midrule
			\texttt{theorems} & OFF & Enables theorem environments (theorem, lemma, etc.) \\
			\texttt{algorithms} & OFF & Enables algorithm and pseudocode support \\
			\texttt{minimal} & OFF & Loads only core math features, no graphics \\
			\bottomrule
		\end{tabularx}
		\caption{Package options in textool v4.0}
	\end{table}
	
	\subsection{Package Philosophy}
	
	The package follows these design principles:
	\begin{enumerate}
		\item \textbf{Safety First}: Opt-in model prevents conflicts with user definitions
		\item \textbf{Consistency}: All symbols follow predictable naming patterns
		\item \textbf{Simplicity}: Complex tasks require minimal syntax
		\item \textbf{Compatibility}: Works alongside other standard packages
		\item \textbf{Completeness}: Provides comprehensive symbol coverage
		\item \textbf{Robustness}: Conditional loading prevents redefinition errors
	\end{enumerate}
	
	% ============================================
	\section{Mathematical Symbols}
	% ============================================
	
	\subsection{Symbol Naming Convention}
	
	TexTool provides systematic access to mathematical symbols using consistent prefixes:
	
	\begin{table}[h]
		\centering
		\begin{tabular}{lll}
			\toprule
			\textbf{Prefix} & \textbf{Meaning} & \textbf{Example} \\
			\midrule
			\texttt{mb} & Blackboard bold & \texttt{\textbackslash mbR} $\rightarrow$ $\mbR$ \\
			\texttt{ccal} & Calligraphic & \texttt{\textbackslash ccalF} $\rightarrow$ $\ccalF$ \\
			\texttt{bb} & Bold & \texttt{\textbackslash bbx} $\rightarrow$ $\bbx$ \\
			\texttt{bbar} & Bar over symbol & \texttt{\textbackslash bbarx} $\rightarrow$ $\bbarx$ \\
			\texttt{hhat} & Hat over symbol & \texttt{\textbackslash hhatx} $\rightarrow$ $\hhatx$ \\
			\texttt{td} & Tilde over symbol & \texttt{\textbackslash tdx} $\rightarrow$ $\tdx$ \\
			\bottomrule
		\end{tabular}
		\caption{Symbol prefix conventions}
	\end{table}
	
	\subsection{Blackboard Bold Letters}
	
	Used primarily for number sets and spaces:
	
	\begin{example}
		\begin{lstlisting}
			$\mbR$ % Real numbers
			$\mbC$ % Complex numbers
			$\mbN$ % Natural numbers
			$\mbZ$ % Integers
			$\mbQ$ % Rational numbers
			$\mbP$ % Probability space
		\end{lstlisting}
	\end{example}
	
	Result: $\mbR$, $\mbC$, $\mbN$, $\mbZ$, $\mbQ$, $\mbP$
	
	\subsection{Calligraphic Letters}
	
	Used for spaces, algebras, and special sets:
	
	\begin{example}
		\begin{lstlisting}
			$\ccalF$ % Sigma-algebra
			$\ccalH$ % Hilbert space
			$\ccalL$ % Linear operators
			$\ccalM$ % Manifold
		\end{lstlisting}
	\end{example}
	
	Result: $\ccalF$, $\ccalH$, $\ccalL$, $\ccalM$
	
	\subsection{Bold Symbols}
	
	For vectors and matrices:
	
	\begin{example}
		\begin{lstlisting}
			$\bbx \in \mbR^n$ % Vector x
			$\bbA \bbx = \bbb$ % Matrix equation
			$\bbalpha, \bbbeta, \bbgamma$ % Bold Greek
		\end{lstlisting}
	\end{example}
	
	Result: $\bbx \in \mbR^n$, $\bbA \bbx = \bbb$, $\bbalpha, \bbbeta, \bbgamma$
	
	\subsection{Decorated Symbols}
	
	\begin{table}[h]
		\centering
		\begin{tabularx}{\textwidth}{lXl}
			\toprule
			\textbf{Command} & \textbf{Description} & \textbf{Output} \\
			\midrule
			\texttt{\textbackslash bbarX} & Bar over X & $\bbarX$ \\
			\texttt{\textbackslash hhatX} & Hat over X & $\hhatX$ \\
			\texttt{\textbackslash tdX} & Tilde over X & $\tdX$ \\
			\texttt{\textbackslash bbarx} & Bar over x & $\bbarx$ \\
			\texttt{\textbackslash hhatx} & Hat over x & $\hhatx$ \\
			\texttt{\textbackslash tdx} & Tilde over x & $\tdx$ \\
			\texttt{\textbackslash bbargamma} & Bar over gamma & $\bbargamma$ \\
			\texttt{\textbackslash hhattheta} & Hat over theta & $\hhattheta$ \\
			\texttt{\textbackslash tdsigma} & Tilde over sigma & $\tdsigma$ \\
			\bottomrule
		\end{tabularx}
		\caption{Decorated symbol examples}
	\end{table}
	
	% ============================================
	\section{Probability and Statistics}
	% ============================================
	
	\subsection{Expectation Operators}
	
	TexTool provides multiple expectation notations:
	
	\begin{example}
		\begin{lstlisting}
			$\E{X}$          % E[X]
			$\Ec{X}$         % E(X) with parentheses
			$\Eb{X^2}$       % E with big brackets
			$\EB{X^2}$       % E with Big brackets
			$\EE{Y}{X|Y}$    % E_Y[X|Y] conditional expectation
		\end{lstlisting}
	\end{example}
	
	Result: $\E{X}$, $\Ec{X}$, $\Eb{X^2}$, $\EB{X^2}$, $\EE{Y}{X|Y}$
	
	\subsection{Probability Operators}
	
	\begin{example}
		\begin{lstlisting}
			$\Pr{X > 0}$     % P[X > 0]
			$\Prc{A \cup B}$ % P(A union B) with parentheses
			$\Pb{X = k}$     % P with big brackets
			$\PP{X}{Y|X}$    % P_X[Y|X] with subscript
		\end{lstlisting}
	\end{example}
	
	Result: $\Pr{X > 0}$, $\Prc{A \cup B}$, $\Pb{X = k}$, $\PP{X}{Y|X}$
	
	\subsection{Statistical Functions}
	
	\begin{example}
		\begin{lstlisting}
			$\Var{X}$              % Variance
			$\Covp{X}{Y}$          % Covariance
			$\ind{x > 0}$          % Indicator function
			$\Ind{A}$              % Alternative indicator
			$\Ent{X}$              % Entropy H[X]
			$\KL{P}{Q}$            % KL divergence
			$\MI{X}{Y}$            % Mutual information
		\end{lstlisting}
	\end{example}
	
	Result: $\Var{X}$, $\Covp{X}{Y}$, $\ind{x > 0}$, $\Ind{A}$, $\Ent{X}$, $\KL{P}{Q}$, $\MI{X}{Y}$
	
	% ============================================
	\section{Figure Management}
	% ============================================
	
	\begin{important}
		Figure commands are available by default unless you use the \texttt{minimal} option.
	\end{important}
	
	\subsection{Single Figures}
	
	TexTool simplifies figure insertion with brace-based syntax:
	
	\begin{example}
		\begin{lstlisting}
			% Basic figure (80% width)
			\fig{{image.png},{Caption text},{label}}
			
			% Custom width figure
			\figw[0.6]{{image.png},{Caption text},{label}}
			
			% Exact placement figure
			\figH{{image.png},{Caption text},{label}}
			
			% Figure without caption
			\fignc{{image.png},{label}}
		\end{lstlisting}
	\end{example}
	
	\begin{tip}
		Labels are automatically prefixed with \texttt{fig:}, so you reference them as \texttt{\textbackslash ref\{fig:label\}}
	\end{tip}
	
	\subsection{Multi-Figure Grids}
	
	TexTool supports grid layouts from 1$\times$2 up to 4$\times$4:
	
	\subsubsection{2$\times$2 Grid Example}
	
	\begin{lstlisting}
		\mcfigTwoByTwo{{Overall caption},{main-label}}
		{{img1.png},{Caption 1},{label1}}
		{{img2.png},{Caption 2},{label2}}
		{{img3.png},{Caption 3},{label3}}
		{{img4.png},{Caption 4},{label4}}
	\end{lstlisting}
	
	\subsubsection{Available Grid Commands}
	
	\begin{table}[h]
		\centering
		\begin{tabular}{ll}
			\toprule
			\textbf{Command} & \textbf{Grid Layout} \\
			\midrule
			\texttt{\textbackslash mcfigOneByTwo} & 1$\times$2 (2 images) \\
			\texttt{\textbackslash mcfigTwoByTwo} & 2$\times$2 (4 images) \\
			\bottomrule
		\end{tabular}
		\caption{Multi-figure grid commands (partial list)}
	\end{table}
	
	\begin{tip}
		The package provides many more grid layouts. Check the source code for the complete list of available \texttt{mcfig} commands.
	\end{tip}
	
	\subsection{Spacing Control}
	
	Adjust spacing in multi-figure layouts:
	
	\begin{lstlisting}
		\setSubfigTopSkip{10pt}    % Space above figures
		\setSubfigBottomSkip{5pt}  % Space between rows
		\setSubfigCapSkip{3pt}     % Space below captions
	\end{lstlisting}
	
	% ============================================
	\section{Theorem Environments (Optional)}
	% ============================================
	
	\begin{warning}
		\textbf{Theorem environments are OFF by default in v4.0!}\\
		You must explicitly enable them with \texttt{\textbackslash usepackage[theorems]\{textool\}}
	\end{warning}
	
	\subsection{Enabling Theorems}
	
	To use theorem environments, you must load the package with the \texttt{theorems} option:
	
	\begin{example}
		\begin{lstlisting}
			% Enable theorems
			\usepackage[theorems]{textool}
			
			% Or with algorithms too
			\usepackage[theorems,algorithms]{textool}
		\end{lstlisting}
	\end{example}
	
	\subsection{Available Environments}
	
	When enabled, TexTool provides pre-styled theorem environments:
	
	\begin{example}
		\begin{lstlisting}
			\begin{theorem}
				Every continuous function on a compact set is uniformly continuous.
			\end{theorem}
			
			\begin{lemma}
				If $f$ is differentiable, then $f$ is continuous.
			\end{lemma}
			
			\begin{definition}
				A set $S$ is \emph{compact} if every open cover has a finite subcover.
			\end{definition}
		\end{lstlisting}
	\end{example}
	
	\subsection{Complete List of Theorem Environments}
	
	\begin{table}[h]
		\centering
		\begin{tabularx}{\textwidth}{lX}
			\toprule
			\textbf{Environment} & \textbf{Purpose} \\
			\midrule
			\texttt{theorem} & Major results \\
			\texttt{lemma} & Supporting results \\
			\texttt{proposition} & Standalone statements \\
			\texttt{corollary} & Direct consequences \\
			\texttt{definition} & Formal definitions \\
			\texttt{remark} & Clarifying comments \\
			\texttt{assumption} & Stated assumptions \\
			\texttt{observation} & Notable observations \\
			\texttt{fact} & Known facts \\
			\texttt{property} & Mathematical properties \\
			\texttt{test} & Test cases \\
			\bottomrule
		\end{tabularx}
		\caption{Theorem-like environments (when enabled)}
	\end{table}
	
	\subsection{Custom Proof Environment}
	
	The \texttt{myproof} environment is always available (doesn't require the \texttt{theorems} option):
	
	\begin{example}
		\begin{lstlisting}
			\begin{myproof}
				Let $x \in S$. By assumption...
				Therefore, the statement holds.
			\end{myproof}
			
			\begin{myproof}[of Theorem 3.1]
				Using the previous lemma...
			\end{myproof}
		\end{lstlisting}
	\end{example}
	
	% ============================================
	\section{Algorithm Support (Optional)}
	% ============================================
	
	\begin{warning}
		\textbf{Algorithm support is OFF by default in v4.0!}\\
		You must explicitly enable it with \texttt{\textbackslash usepackage[algorithms]\{textool\}}
	\end{warning}
	
	\subsection{Enabling Algorithms}
	
	To use algorithm environments, load the package with the \texttt{algorithms} option:
	
	\begin{example}
		\begin{lstlisting}
			% Enable algorithms
			\usepackage[algorithms]{textool}
			
			% The package will load algpseudocode and algorithm
			% unless algorithmic is already loaded
		\end{lstlisting}
	\end{example}
	
	\subsection{Compatibility Notes}
	
	\begin{important}
		The package intelligently detects existing algorithm packages:
		\begin{itemize}
			\item If \texttt{algorithmic} is already loaded, \texttt{algpseudocode} will be skipped
			\item If neither is loaded, \texttt{algpseudocode} is preferred
			\item The \texttt{algorithm} float environment is loaded if available
		\end{itemize}
	\end{important}
	
	% ============================================
	\section{Equation Environments}
	% ============================================
	
	\subsection{Standard Equation Helpers}
	
	\begin{example}
		\begin{lstlisting}
			% Aligned equation with number
			\feq{
				f(x) &= x^2 + 2x + 1 \\
				&= (x + 1)^2
			}
			
			% Aligned equation without number
			\nfeq{
				g(x) &= \sin(x) + \cos(x) \\
				&= \sqrt{2}\sin(x + \pi/4)
			}
		\end{lstlisting}
	\end{example}
	
	\subsection{Slide Equation Environments}
	
	For presentations and slides:
	
	\begin{table}[h]
		\centering
		\begin{tabularx}{\textwidth}{lX}
			\toprule
			\textbf{Environment} & \textbf{Description} \\
			\midrule
			\texttt{slideeq} & Regular equation for slides \\
			\texttt{nslideeq} & Equation without numbering \\
			\texttt{sslideeq} & Small size equation \\
			\texttt{fslideeq} & Footnote size equation \\
			\texttt{slidealign} & Aligned equations \\
			\texttt{nslidealign} & Aligned without numbering \\
			\texttt{sslidealign} & Small aligned equations \\
			\texttt{fslidealign} & Footnote size aligned \\
			\bottomrule
		\end{tabularx}
	\end{table}
	
	% ============================================
	\section{Color and Formatting}
	% ============================================
	
	\begin{important}
		Colors and formatting commands are not available in \texttt{minimal} mode.
	\end{important}
	
	\subsection{Predefined Colors}
	
	TexTool defines several colors for highlighting:
	
	\begin{example}
		\begin{lstlisting}
			\red{Error message}
			\blue{Information}
			\green{Success}
			\grey{Disabled text}
			\highlight{Important}
			\todo{Remember this}
		\end{lstlisting}
	\end{example}
	
	\subsection{Arrow Bullets}
	
	For itemized lists with arrows:
	
	\begin{example}
		\begin{lstlisting}
			\begin{itemize}
				\arritem First point
				\ai Second point (short form)
				\item[\darrbullet] Two-way relationship
			\end{itemize}
		\end{lstlisting}
	\end{example}
	
	% ============================================
	\section{List Environments}
	% ============================================
	
	\subsection{Custom List Environment}
	
	The \texttt{mylist} environment provides controlled spacing:
	
	\begin{example}
		\begin{lstlisting}
			\begin{mylist}
				\item First item with proper spacing
				\item Second item
				\item Third item
			\end{mylist}
		\end{lstlisting}
	\end{example}
	
	\subsection{Exercise Environment}
	
	For homework and exercises:
	
	\begin{example}
		\begin{lstlisting}
			\exercise{Prove the fundamental theorem of calculus}
			
			\exercisepart{State the theorem}
			The theorem states that...
			
			\exercisepart{Provide the proof}
			We begin by considering...
			
			\exercisepart{Give an example}
			Consider $f(x) = x^2$...
		\end{lstlisting}
	\end{example}
	
	% ============================================
	\section{Text Commands}
	% ============================================
	
	\subsection{Mathematical Text}
	
	TexTool provides text commands for use in math mode:
	
	\begin{table}[h]
		\centering
		\begin{tabular}{ll}
			\toprule
			\textbf{Command} & \textbf{Output} \\
			\midrule
			\texttt{\textbackslash rank} & $\rank$ \\
			\texttt{\textbackslash diag} & $\diag$ \\
			\texttt{\textbackslash tr} & $\tr$ (trace) \\
			\texttt{\textbackslash st} & $\st$ (subject to) \\
			\texttt{\textbackslash sign} & $\sign$ \\
			\texttt{\textbackslash argmax} & $\argmax$ \\
			\texttt{\textbackslash argmin} & $\argmin$ \\
			\bottomrule
		\end{tabular}
	\end{table}
	
	\subsection{Section Commands}
	
	Custom section formatting:
	
	\begin{lstlisting}
		\mysubsection{Section Title}
		\mysubsubsection{Subsection Title}
	\end{lstlisting}
	
	% ============================================
	\section{Advanced Examples}
	% ============================================
	
	\subsection{Complete Mathematical Document}
	
	\begin{example}
		\begin{lstlisting}
			\documentclass{article}
			\usepackage[theorems]{textool}  % Enable theorems
			
			\begin{document}
				
				\begin{theorem}
					Let $\bbX \in \mbR^{n \times m}$ be a random matrix with 
					$\E{\bbX} = \bbM$ and $\Var{\bbX_{ij}} = \sigma^2$.
					Then $\bbarX = \frac{1}{nm}\sum_{i,j} \bbX_{ij}$ 
					converges to $\tr(\bbM)/nm$ as $n,m \to \infty$.
				\end{theorem}
				
				\begin{myproof}
					By the law of large numbers, we have
					\feq{
						\Pr{\left|\bbarX - \E{\bbarX}\right| > \epsilon} 
						&\leq \frac{\Var{\bbarX}}{\epsilon^2} \\
						&= \frac{\sigma^2}{nm\epsilon^2} \to 0
					}
					as $n,m \to \infty$.
				\end{myproof}
				
			\end{document}
		\end{lstlisting}
	\end{example}
	
	\subsection{Using Different Package Options}
	
	\begin{example}
		\begin{lstlisting}
			% For a math-heavy document without figures
			\documentclass{article}
			\usepackage[minimal]{textool}
			
			% For a document with theorems but no algorithms
			\documentclass{article}
			\usepackage[theorems]{textool}
			
			% For a complete document with all features
			\documentclass{article}
			\usepackage[theorems,algorithms]{textool}
			
			% When using other theorem packages
			\documentclass{article}
			\usepackage{amsthm}  % Load your theorem package first
			\usepackage{textool} % textool won't redefine theorems
		\end{lstlisting}
	\end{example}
	
	% ============================================
	\section{Troubleshooting}
	% ============================================
	
	\subsection{Common Issues}
	
	\begin{warning}
		\textbf{Missing Theorem Environments}: In v4.0, theorems are OFF by default. If you get "undefined environment" errors, add the \texttt{theorems} option: \texttt{\textbackslash usepackage[theorems]\{textool\}}
	\end{warning}
	
	\begin{warning}
		\textbf{Algorithm Package Conflicts}: The package checks for existing algorithm packages. If you need \texttt{algorithmic} instead of \texttt{algpseudocode}, load it before textool.
	\end{warning}
	
	\begin{warning}
		\textbf{Symbol Conflicts}: If a symbol is already defined by another package, textool will skip that definition. Load textool after other math packages if you prefer its definitions.
	\end{warning}
	
	\begin{warning}
		\textbf{Tabularx Compatibility}: Version 4.0 loads the \texttt{array} package first to ensure tabularx compatibility.
	\end{warning}
	
	\subsection{Best Practices}
	
	\begin{enumerate}
		\item \textbf{Package Order}: Load \texttt{textool} after basic math packages but before document-specific definitions
		\item \textbf{Choose Options Carefully}: Only enable features you need (theorems, algorithms)
		\item \textbf{Minimal Mode}: Use \texttt{minimal} option for documents without figures
		\item \textbf{Theorem Conflicts}: If using another theorem package, don't enable the \texttt{theorems} option
		\item \textbf{Check Console Output}: The package prints loading information to help debug issues
	\end{enumerate}
	
	\subsection{Package Loading Messages}
	
	When loaded, textool outputs helpful information:
	
	\begin{lstlisting}
		===========================================
		textool v4.0 Robust Edition loaded
		Options: theorems=OFF (default), algorithms=OFF (default)
		Array package loaded for tabularx compatibility
		For basic use: \usepackage{textool}
		With theorems: \usepackage[theorems]{textool}
		===========================================
	\end{lstlisting}
	
	% ============================================
	\section{Migration Guide from Earlier Versions}
	% ============================================
	
	\subsection{From v1.x to v4.0}
	
	If you're upgrading from version 1.x, note these important changes:
	
	\begin{important}
		\textbf{Breaking Changes:}
		\begin{itemize}
			\item Theorems are now OFF by default - add \texttt{theorems} option
			\item Algorithms are now OFF by default - add \texttt{algorithms} option
			\item Some commands are now conditional to avoid conflicts
		\end{itemize}
	\end{important}
	
	\begin{table}[h]
		\centering
		\begin{tabularx}{\textwidth}{XX}
			\toprule
			\textbf{Old (v1.x)} & \textbf{New (v4.0)} \\
			\midrule
			\texttt{\textbackslash usepackage\{textool\}} 
			(got everything) & 
			\texttt{\textbackslash usepackage[theorems,algorithms]\{textool\}}
			(explicit opt-in) \\
			\midrule
			Theorems always loaded & 
			Theorems only with \texttt{theorems} option \\
			\midrule
			Algorithms always loaded & 
			Algorithms only with \texttt{algorithms} option \\
			\bottomrule
		\end{tabularx}
		\caption{Migration from v1.x to v4.0}
	\end{table}
	
	% ============================================
	\section{Quick Reference Card}
	% ============================================
	
	\subsection{Most Used Commands}
	
	\begin{table}[h]
		\centering
		\small
		\begin{tabularx}{\textwidth}{lX}
			\toprule
			\textbf{Category} & \textbf{Common Commands} \\
			\midrule
			\textbf{Package Loading} & \texttt{\textbackslash usepackage[options]\{textool\}} \\
			\textbf{Options} & \texttt{theorems}, \texttt{algorithms}, \texttt{minimal} \\
			\textbf{Sets} & \texttt{\textbackslash mbR, \textbackslash mbC, \textbackslash mbN, \textbackslash mbZ} \\
			\textbf{Vectors} & \texttt{\textbackslash bbx, \textbackslash bby, \textbackslash bbA} \\
			\textbf{Statistics} & \texttt{\textbackslash E\{X\}, \textbackslash Var\{X\}, \textbackslash Pr\{A\}} \\
			\textbf{Figures} & \texttt{\textbackslash fig\{\{file\},\{caption\},\{label\}\}} \\
			\textbf{Theorems*} & \texttt{\textbackslash begin\{theorem\}...\textbackslash end\{theorem\}} \\
			\textbf{Greek} & \texttt{\textbackslash bbalpha, \textbackslash tdbeta, \textbackslash bbargamma} \\
			\bottomrule
		\end{tabularx}
		\caption{Quick reference (* = requires \texttt{theorems} option)}
	\end{table}
	
	% ============================================
	\appendix
	\section{Complete Symbol Tables}
	% ============================================
	
	\subsection{All Blackboard Bold Symbols}
	
	\begin{center}
		\begin{tabular}{cccccc}
			\toprule
			$\mbA$ & $\mbB$ & $\mbC$ & $\mbD$ & $\mbE$ & $\mbF$ \\
			$\mbG$ & $\mbH$ & $\mbI$ & $\mbJ$ & $\mbK$ & $\mbL$ \\
			$\mbM$ & $\mbN$ & $\mbO$ & $\mbP$ & $\mbQ$ & $\mbR$ \\
			$\mbS$ & $\mbT$ & $\mbU$ & $\mbV$ & $\mbW$ & $\mbX$ \\
			$\mbY$ & $\mbZ$ & & & & \\
			\bottomrule
		\end{tabular}
	\end{center}
	
	\subsection{All Bold Symbols}
	
	\textbf{Uppercase:}
	\begin{center}
		\begin{tabular}{cccccc}
			$\bbA$ & $\bbB$ & $\bbC$ & $\bbD$ & $\bbE$ & $\bbF$ \\
			$\bbG$ & $\bbH$ & $\bbI$ & $\bbJ$ & $\bbK$ & $\bbL$ \\
			$\bbM$ & $\bbN$ & $\bbO$ & $\bbP$ & $\bbQ$ & $\bbR$ \\
			$\bbS$ & $\bbT$ & $\bbU$ & $\bbV$ & $\bbW$ & $\bbX$ \\
			$\bbY$ & $\bbZ$ & & & & \\
		\end{tabular}
	\end{center}
	
	\textbf{Lowercase:}
	\begin{center}
		\begin{tabular}{cccccc}
			$\bba$ & $\bbb$ & $\bbc$ & $\bbd$ & $\bbe$ & $\bbf$ \\
			$\bbg$ & $\bbh$ & $\bbi$ & $\bbj$ & $\bbk$ & $\bbl$ \\
			$\bbm$ & $\bbn$ & $\bbo$ & $\bbp$ & $\bbq$ & $\bbr$ \\
			$\bbs$ & $\bbt$ & $\bbu$ & $\bbv$ & $\bbw$ & $\bbx$ \\
			$\bby$ & $\bbz$ & & & & \\
		\end{tabular}
	\end{center}
	
	% ============================================
	\section{Version History}
	% ============================================
	
	\begin{itemize}
		\item \textbf{Version 1.0}: Initial release
		\item \textbf{Version 1.1}: Added compatibility checks, made all definitions conditional
		\item \textbf{Version 2.x}: Added conflict resolution mechanisms
		\item \textbf{Version 3.0}: Changed to opt-in model for safety
		\item \textbf{Version 4.0}: 
		\begin{itemize}
			\item Fixed tabularx compatibility with array package
			\item Restored all features from original
			\item Enhanced robustness
			\item Improved package option system
			\item Better conflict detection
			\item Comprehensive loading messages
		\end{itemize}
	\end{itemize}
	
	% ============================================
	\section{License and Contact}
	% ============================================
	
	This package is provided as-is for academic use. For questions, bug reports, or feature requests, please contact the author at the email address provided.
	
	The package development follows the principle of maintaining backward compatibility while improving robustness and preventing conflicts with other commonly used packages.
	
	\vspace{2cm}
	\begin{center}
		\rule{0.5\textwidth}{0.5pt}\\[1em]
		\textit{End of TexTool Package Handbook v4.0}
	\end{center}
	
\end{document}